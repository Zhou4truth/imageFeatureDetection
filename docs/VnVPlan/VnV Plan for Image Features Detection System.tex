\documentclass[12pt, titlepage]{article}

\usepackage{booktabs}
\usepackage{tabularx}
\usepackage{hyperref}
\hypersetup{
    colorlinks,
    citecolor=blue,
    filecolor=black,
    linkcolor=red,
    urlcolor=blue
}
\usepackage[round]{natbib}



\begin{document}

\title{Project Title: System Verification and Validation Plan for\\ \progname{Image Features Detection System}} 
\author{Gaofeng Zhou}
\date{Feb.15,2024}
	
\maketitle

\pagenumbering{roman}

\section*{Revision History}

\begin{tabularx}{\textwidth}{p{3cm}p{2cm}X}
\toprule {\bf Date} & {\bf Version} & {\bf Notes}\\
\midrule
Feb.18, 2024 & 1.0 & Original plan\\

\bottomrule
\end{tabularx}

~\\


\newpage

\tableofcontents

\newpage


\newpage

\section{Symbols, Abbreviations, and Acronyms}

\renewcommand{\arraystretch}{1.2}
\begin{tabular}{l l} 
  \toprule		
  \textbf{symbol} & \textbf{description}\\
  \midrule 
  T & Test\\
  I & Image\\
  
G(x)& Gaussian Transform or Guassian Filtering\\
L(x)& Laplacian Transform\\
DoG &Differential of Gaussian Transform of Gaussian Transform\\
LoG &Laplacian Transform of Gaussian Transform of Gaussian Transform\\
SIFT& Scale Invariant Feature Transform\\
SRS &Software Requirements Specification\\
  
  \bottomrule
\end{tabular}\\



\newpage

\pagenumbering{arabic}

\section{General Information}

\subsection{Summary}

   The Image Features Detection System  will be tested according to this plan here. This system can be seen as a preliminary step before doing 2d image stitching or 3d image reconstruction. It will use two image as the inputs, and produce image feature points such as corner points, Edge Points, Contour Points, Gradients, and SIFT Feature Points. All these feature points will be used to have first visual judgement before further image processing.


\subsection{Objectives}


The objectives of this system is to use two different images as inputs and then show the feature points of the two images as the outputs generated by those image processing algorithms. There would be a GUI display setting for this system. On this GUI, there would be different buttons which indicate different functions to do image features detection. There should be two windows to show the input images and two windows to their features points on this GUI. With each click of these buttons, the result will be displayed on the GUI.\\

Besides those basic objectives of this system, it also should suffice the requirements of accuracy and usability. The accuracy should be in accordance with verified standards as which here we use MATLAB. The usability means the layout of this system should be user friendly. And the processing time should be in a high efficiency.

\subsection{Relevant Documentation}
SRS document
\href{https://github.com/Zhou4truth/imageFeatureDetection/blob/main/docs/SRS/Software%20Requriement%20Specifications.pdf}{SRS}



\section{Plan}

The plan to do verification and validation will consist of SRS VnV plan, Unit VnV plan, Integration and system VnV plan.

\subsection{Verification and Validation Team}

\begin{tabular}{l l} 
  \toprule		
  \textbf{Name} & \textbf{Role}\\
  \midrule 
  Valerie Vreugdenhil& Domain Expert and Primary Reviewer\\
  Seyed Ali Mousavi& Second reviewer for SRS\\
  Hossain Al Jubair& Second reviewer for VnV plan\\
  Xinyu Ma& Second reviewer for MG +MIS \\

  
  \bottomrule
\end{tabular}\\

\subsection{SRS Verification Plan}

For the SRS verification plan, we will apply the experts review as the main method to do this.

\subsection{Design Verification Plan}

For this part, we will use this checklist as below to do this VnV process:
1. Is this design theoretically workable?
2. Is this design technically feasible?
3. Is this design the best way to solve this problem?
4. What do we need to complete this system as the design plan required? Is the cost worthy?


\subsection{Verification and Validation Plan Verification Plan}

We will apply expert review to do this plan.


\subsection{Implementation Verification Plan}

For the code part, we will use copilot to the codes check.\\
For the algorithm part, we will use MATLAB to verify.\\
For the GUI and integration part, we will use Qt test to achieve the result.\\


\subsection{Automated Testing and Verification Tools}

Here we may use github actions to do the CI/CD monitoring as we will let this project open sourced on github.


\subsection{Software Validation Plan}
   Experts reviews will be considered here. and the algorithm part will be compared with MATLAB.

\section{System Test Description}
The whole system test will be processed in two parts. One is the functional requirements test, and the other is the non-function requirements test.\\
	
\subsection{Tests for Functional Requirements}
The functional requirements consist of GUI test, buttons function test, and display test.
\subsubsection{GUI test}

GUI here should be generate with Qt platform. and the GUI with integrated functions inside should be working accurately and well organized.
		
\paragraph{ Button test}

\begin{enumerate}

\item{test-1-import button\\}

Control: Manual 
					
Initial State: first step\\
					
Input: two different images with good precondition as required.\\
					
Output: respect image displayed on two windows after the "import" button clicked.\\


					
How test will be performed: This test will be performed manually with the import button.
					
\item{test-2- Grey button\\}

Control: Manual 
					
Initial State: After image transformed into grey images
					
Input: two images as the test 1.
					
Output: There would be two grey images displayed on the GUI windows which were transformed from the input images after the "Grey" button clicked.


How test will be performed:  After the RGB images were imported successfully, we click the "grey" button. Then there would be two grey images displayed on the GUI windows if this button worked well.
\item{test-3- Corner button\\}

Control: Manual 
					
Initial State: After image imported
					
Input: two images as the test 1.
					
Output: There would be corner points drawn in the two images displayed on the GUI windows after the "corner" button clicked.


How test will be performed:  After the RGB images were transformed into grey images successfully, we click the "corner" button. Then there would be corner points appeared on the two grey images displayed on the GUI windows if this button worked well.
\item{test-4- Contour button\\}

Control: Manual 
					
Initial State: After images were transformed into grey images.
					
Input: two images as the test 1.
					
Output: There would be contour points drawn in the two images displayed on the GUI windows after the "contour" button clicked.


How test will be performed:  After the RGB images were transformed into grey images successfully, we click the "contour" button. Then there would be contour points appeared on the two grey images displayed on the GUI windows if this button worked well.
\item{test-5- Edge button\\}

Control: Manual 
					
Initial State: After images were transformed into grey images.
					
Input: two images as the test 1.
					
Output: There would be Edge points drawn in the two images displayed on the GUI windows after the "Edge" button clicked.


How test will be performed:  After the RGB images were transformed into grey images successfully, we click the "Edge" button. Then there would be edge points appeared on the two grey images displayed on the GUI windows if this button worked well.
\item{test-6- SIFT button\\}

Control: Manual 
					
Initial State: After images were transformed into grey images.
					
Input: two images as the test 1.
					
Output: There would be the first 50 best SIFT feature points drawn in the two images displayed on the GUI windows after the "SIFT" button clicked.


How test will be performed:  After the RGB images were transformed into grey images successfully, we click the "SIFT" button. Then there would be SIFT feature points appeared on the two grey images displayed on the GUI windows if this button worked well.
\end{enumerate}


\subsection{Tests for Nonfunctional Requirements}

Accuracy: we will compare the points we got with this system with the outcome we got with MATLAB.\\\\
Usability: we will do a survey to review the usability of this system.\\

\subsubsection{Accuracy test}
		
\paragraph{Algorithm accuracy verification}

\begin{enumerate}

\item{test-1-Corner points accuracy test\\}

Type: Manual
					
Initial State: 
					
Input/Condition: First best 50 points generated by our system and MATALB respectively.
					
Output/Result: the average absolute error value of this 50 points.
					
How test will be performed: Firstly, we got 50 best edge points with the edge detection algorithm in our system, and then we would get 50 best edge points with edge detection algorithm in MATLAB. After that, we calculate these the differences of these 50 pairs of best edge feature points. We sum them up and then let the summation divided by 50,then we can get the value we needed.
					
\item{test-2-Contour points accuracy test\\}

Type: Manual
					
Initial State: 
					
Input/Condition: First best 50 points generated by our system and MATALB respectively.
					
Output/Result: the average absolute error value of this 50 points.
					
How test will be performed: Firstly, we got 50 best contour points with the edge detection algorithm in our system, and then we would get 50 best contour points with edge detection algorithm in MATLAB. After that, we calculate these the differences of these 50 pairs of best contour feature points. We sum them up and then let the summation divided by 50,then we can get the value we needed.
\item{test-3-Corner points accuracy test\\}

Type: Manual
					
Initial State: 
					
Input/Condition: First best 50 points generated by our system and MATALB respectively.
					
Output/Result: the average absolute error value of this 50 points.
					
How test will be performed: Firstly, we got 50 best corner points with the corner detection algorithm in our system, and then we would get 50 best corner points with corner detection algorithm in MATLAB. After that, we calculate these the differences of these 50 pairs of best corner feature points. We sum them up and then let the summation divided by 50,then we can get the value we needed.
\item{test-4-SIFT points accuracy test\\}

Type: Manual
					
Initial State: 
					
Input/Condition: First best 50 points generated by our system and MATALB respectively.
					
Output/Result: the average absolute error value of this 50 points.
					
How test will be performed: Firstly, we got 50 best SIFT points with the edge detection algorithm in our system, and then we would get 50 best SIFT points with SIFT detection algorithm in MATLAB. After that, we calculate these the differences of these 50 pairs of best SIFT feature points. We sum them up and then let the summation divided by 50,then we can get the value we needed.
\end{enumerate}

\subsubsection{Usability survey}

We will do a usability survey with a questionnaire as the appendix showed.

\subsection{Traceability Between Test Cases and Requirements}

\begin{table}[h]
\centering
\caption{Traceability Matrix}
\label{tab:traceabilityMatrix}
\begin{tabular}{@{}lccccc@{}}
\toprule
 & \textbf{Test 1} & \textbf{Test 2} & \textbf{Test 3} & \textbf{Test 4} &\textbf{Test 5}\\ \midrule
\textbf{Images Imported} & \checkmark &  & \checkmark & \checkmark \\
\textbf{Images grey transformed} &  & \checkmark &  & \checkmark &\checkmark \\
\end{tabular}
\end{table}


\newpage

				
\bibliographystyle{plainnat}

\bibliography{../../refs/References}
Dr.Spencer.Smith, lecture of CAS 741,https://gitlab.cas.mcmaster.ca/smiths/cas741/-/tree/master/Lectures/L09_VerificationAndValidation?ref_type=heads\\
Gaofeng Zhou, SRS  document, https://github.com/Zhou4truth/imageFeatureDetection/blob/main/docs/SRS/Software%20Requriement%20Specifications.pdf 

\newpage

\section{Appendix}

\subsection{Usability Survey Questions?}

1. Do you like the GUI layout of this system? Rate it in the range of 0-10, which 10 stands for best.\\
2. Does the display windows look good of this system? Rate it in the range of 0-10, which 10 stands for best.\\
3. Do you like the designs of these buttons of this system? Rate it in the range of 0-10, which 10 stands for best.\\
4. Do you like the way with which these feature points displayed of this system? Rate it in the range of 0-10, which 10 stands for best.\\
5. If you have any advice on this system, please write in the blank.\\

\newpage{}



\end{document}
