% THIS DOCUMENT IS TAILORED TO REQUIREMENTS FOR SCIENTIFIC COMPUTING.  IT SHOULDN'T
% BE USED FOR NON-SCIENTIFIC COMPUTING PROJECTS
\documentclass[12pt]{article}
\usepackage{amsmath, mathtools}
\usepackage{amsfonts}
\usepackage{amssymb}
\usepackage{graphicx}
\usepackage{colortbl}
\usepackage{xr}
\usepackage{hyperref}
\usepackage{longtable}
\usepackage{xfrac}
\usepackage{tabularx}
\usepackage{float}
\usepackage{siunitx}
\usepackage{booktabs}
\usepackage{caption}
\usepackage{pdflscape}
\usepackage{afterpage}

\usepackage[round]{natbib}

%\usepackage{refcheck}
\hypersetup{
    bookmarks=true,         % show bookmarks bar?
      colorlinks=true,       % false: boxed links; true: colored links
    linkcolor=red,          % color of internal links (change box color with linkbordercolor)
    citecolor=green,        % color of links to bibliography
    filecolor=magenta,      % color of file links
    urlcolor=cyan           % color of external links
}

%%%% Common Parts

\newcommand{\progname}{ProgName} % PUT YOUR PROGRAM NAME HERE
\newcommand{\authname}{Team \#, Team Name
\\ Student 1 name
\\ Student 2 name
\\ Student 3 name
\\ Student 4 name} % AUTHOR NAMES                  

\usepackage{hyperref}
    \hypersetup{colorlinks=true, linkcolor=blue, citecolor=blue, filecolor=blue,
                urlcolor=blue, unicode=false}
    \urlstyle{same}
                                


% For easy change of table widths
\newcommand{\colZwidth}{1.0\textwidth}
\newcommand{\colAwidth}{0.13\textwidth}
\newcommand{\colBwidth}{0.82\textwidth}
\newcommand{\colCwidth}{0.1\textwidth}
\newcommand{\colDwidth}{0.05\textwidth}
\newcommand{\colEwidth}{0.8\textwidth}
\newcommand{\colFwidth}{0.17\textwidth}
\newcommand{\colGwidth}{0.5\textwidth}
\newcommand{\colHwidth}{0.28\textwidth}

% Used so that cross-references have a meaningful prefix
\newcounter{defnum} %Definition Number
\newcommand{\dthedefnum}{GD\thedefnum}
\newcommand{\dref}[1]{GD\ref{#1}}
\newcounter{datadefnum} %Datadefinition Number
\newcommand{\ddthedatadefnum}{DD\thedatadefnum}
\newcommand{\ddref}[1]{DD\ref{#1}}
\newcounter{theorynum} %Theory Number
\newcommand{\tthetheorynum}{TM\thetheorynum}
\newcommand{\tref}[1]{TM\ref{#1}}
\newcounter{tablenum} %Table Number
\newcommand{\tbthetablenum}{TB\thetablenum}
\newcommand{\tbref}[1]{TB\ref{#1}}
\newcounter{assumpnum} %Assumption Number
\newcommand{\atheassumpnum}{A\theassumpnum}
\newcommand{\aref}[1]{A\ref{#1}}
\newcounter{goalnum} %Goal Number
\newcommand{\gthegoalnum}{GS\thegoalnum}
\newcommand{\gsref}[1]{GS\ref{#1}}
\newcounter{instnum} %Instance Number
\newcommand{\itheinstnum}{IM\theinstnum}
\newcommand{\iref}[1]{IM\ref{#1}}
\newcounter{reqnum} %Requirement Number
\newcommand{\rthereqnum}{R\thereqnum}
\newcommand{\rref}[1]{R\ref{#1}}
\newcounter{nfrnum} %NFR Number
\newcommand{\rthenfrnum}{NFR\thenfrnum}
\newcommand{\nfrref}[1]{NFR\ref{#1}}
\newcounter{lcnum} %Likely change number
\newcommand{\lthelcnum}{LC\thelcnum}
\newcommand{\lcref}[1]{LC\ref{#1}}

\usepackage{fullpage}

\newcommand{\deftheory}[9][Not Applicable]
{
\newpage
\noindent \rule{\textwidth}{0.5mm}

\paragraph{RefName: } \textbf{#2} \phantomsection 
\label{#2}

\paragraph{Label:} #3

\noindent \rule{\textwidth}{0.5mm}

\paragraph{Equation:}

#4

\paragraph{Description:}

#5

\paragraph{Notes:}

#6

\paragraph{Source:}

#7

\paragraph{Ref.\ By:}

#8

\paragraph{Preconditions for \hyperref[#2]{#2}:}
\label{#2_precond}

#9

\paragraph{Derivation for \hyperref[#2]{#2}:}
\label{#2_deriv}

#1

\noindent \rule{\textwidth}{0.5mm}

}

\begin{document}

\title{Software Requirements Specification for \progname: Image Feature Detection System} 
\author{Gaofeng Zhou}
\date{April 12th,2024}
	
\maketitle

~\newpage

\pagenumbering{roman}

\tableofcontents

~\newpage

\section*{Revision History}

\begin{tabularx}{\textwidth}{p{3cm}p{2cm}X}
\toprule {\bf Date} & {\bf Version} & {\bf Notes}\\
\midrule
February 5 & 1.0 & Initial  revision \\
April 12 & 2.0& Second revision\\
\bottomrule
\end{tabularx}

~\\


~\newpage

\section{Reference Material}
This section records information for easy references.

\subsection{Table of Symbols}

The table that follows summarizes the symbols used in this document along with
their units.  The choice of symbols was made to be consistent with the heat
transfer literature and with existing documentation for solar water heating
systems.  The symbols are listed in alphabetical order.

\renewcommand{\arraystretch}{1.2}
%\noindent \begin{tabularx}{1.0\textwidth}{l l X}
\noindent \begin{longtable*}{l l p{12cm}} \toprule
\textbf{symbol} & \textbf{unit} & \textbf{description}\\
\midrule 
$img_i$ & none & The $i-th$ image.
\\
$M_(i)$ & none & The matrix of $Image_i $\\
$I_(x)$ & none &Image(x)\\
$G_(x)$ &none  &Gaussian Transform or Gaussian Filtering\\

\\ 
\bottomrule
\end{longtable*}

\subsection{Abbreviations and Acronyms}

\renewcommand{\arraystretch}{1.2}
\begin{tabular}{l l} 
  \toprule		
  \textbf{symbol} & \textbf{description}\\
  \midrule 
  A & Assumption\\
  DD & Data Definition\\
  GD & General Definition\\
  GS & Goal Statement\\
  IM & Instance Model\\
  LC & Likely Change\\
  PS & Physical System Description\\
  R & Requirement\\
  SRS & Software Requirements Specification\\
  IFDS & Image Feature Detection System\\
  TM & Theoretical Model\\
  BRISK & Binary Robust Invariant Scale Key-points\\
  \bottomrule
\end{tabular}\\



\newpage

\pagenumbering{arabic}







\section{Introduction}

Image features play a very foundational role in the area of image processing or, in another words,computer vision. Especially, When we decide to try 3d image reconstruction or 2d image stitching.
Some image features can be detected by specific image process algorithms, and  some of these features can be used for distinguishing whether they are suitable for further processing visually.\\
We develop this IFDS system to showcase the typical features of images such as edge points, corner points, BRISK points to make these feature points of images visually sensible to users who are interested in this area. \\
This IFDS system contains functions to do transformation and features detection using classical algorithms from OpenCV library. \\



\subsection{Purpose of Document}

The primary purpose of this document is to  record the requirements and goals of IFDS, processing steps, algorithms selection. This document is also the basis for future development of IFDS.

\subsection{Scope of Requirements} 

In this system, the images used for inputs should be 2d RGB images with the format constrained as JPEG,PNG,BMP,TIFF.\\ And this system will be distributed on Windows OS platform only.

\subsection{Characteristics of Intended Reader} 
Readers of this document should have a basic understanding of image features. At least, they should know what these feature points stand for and how they will be used in future work.\\
\subsection{Organization of Document}

This document provides the goals, theories, definitions, road maps and assumptions.

\section{General System Description}

This section provides general information about the system.  It identifies the interfaces between the system and its environment, describes the user characteristics and lists the system constraints.  


\subsection{System Context}

\begin{figure}[h!]
\begin{center}
    \centering
    \includegraphics[width=0.9\textwidth]{Structure_SRS.jpg}
    \caption{System Context}
    \label{fig_SystemContext}
\end{center}  
\end{figure}


\begin{itemize}
\item User Responsibilities:
\begin{itemize}
\item  Input RGB images with formats as the system required.
\end{itemize}
\item \progname{IFDS} Responsibilities:
\begin{itemize}
\item Import images and transform RGB images into Grayscale images.
\item Detect corner points.
\item Detect edge feature points.
\item Detect BRISK feature points.
\item Display corresponding feature points.
\item Export feature points image.
\end{itemize}
\item External Library Responsibilities:
\begin{itemize}
    \item Provide classical feature detection functions for IFDS.
\end{itemize}
\end{itemize}

\subsection{User Characteristics} \label{SecUserCharacteristics}

 The end users of IFDS should be able to install software on Windows OS. And they can understand what those image features are meant to do.

\subsection{System Constraints}
\begin{itemize}
    \item This IFDS must be able to be run on Windows OS platform with distribution installer.and the Windows version should be updated to be newer than Windows 10.
    \item This IFDS will be developed with C++ and with the C++ libraries in OpenCV.
    \item This IFDS must have a GUI to control the system and display the images and the feature points.
    
\end{itemize}



\section{Specific System Description}

This section first presents the problem description, which gives a high-level
view of the problem to be solved.  This is followed by the solution characteristics
specification, which presents the assumptions, theories, definitions and finally
the instance models. 

\subsection{Problem Description} \label{Sec_pd}

\progname{ IFDS } is intended to solve\plt{detecting image feature points within one system to make them visually sensible.}

\subsubsection{Terminology and  Definitions}

\begin{itemize}
\item Feature points: a feature is a piece of information about the content of an image; typically about whether a certain region of the image has certain properties.
\item Corner points: In object detection, the intersection of two perpendicularly intersecting edges\cite{ZhangY2021}.
\item Edge points:The position of pixels in an image where certain quantities or properties (such as brightness) change rapidly.The pixel position in the image where the gradient is greater than a given threshold\cite{ZhangY2021}.
\item BRISK feature points:a novel method for keypoint detection, description and matching\cite{Leutenegger2011}.It can be used to replace the SIFT algorithm to do the Scale Invariant Feature Transformation.

\end{itemize}

\subsubsection{Goal Statements}


The goal statements are:

\begin{itemize}

\item GS1: Given a RGB image, this system can transfer the RGB image into Grayscale image first. and then it can do corner detection, edge detection and BRISK feature points detection.
\item GS2: Given an imported image by users, this system would display this image with its original image and grayscale image on a GUI, and the feature points should be displayed on the grayscale image.
\item GS3: This system should be able to save those feature points images.

\end{itemize}

\subsection{Solution Characteristics Specification}
The instance models that govern \progname{IFDS} are presented in
Subsection~\ref{sec_instance}.  The information to understand the meaning of the instance models and their derivation is also presented, so that the instance models can be verified.

\subsubsection{Assumptions} \label{sec_assumpt}

This section simplifies the original problem and helps in developing the
theoretical model by filling in the missing information for the physical system.
The numbers given in the square brackets refer to the theoretical model [TM],
general definition [GD], data definition [DD], instance model [IM], or likely
change [LC], in which the respective assumption is used.

\begin{itemize}

\item A1: Input images must be RGB images with formats limited among "JPEG, PNG, BMP, TIFF".
\item A2  This system Only detects corner points, edge points and BRISK points.
\item A3: The system will be only installed on Windows OS platform.

\end{itemize}

\subsubsection{Theoretical Models}\label{sec_theoretical}
The IFDS system uses typical image processing methods to do feature points detection. Mainly, it derived from the theory of matrix and gradients operation.


~\newline

\noindent
\deftheory
% #2 refname of theory
{Gradients Calculation}
% #3 label
{Gradients Calculation}
% #4 equation
{
  
$\nabla I_(x_)=$ \begin{bmatrix}
\frac{\partial I_(x_)}{\partial x}, 
\frac{\partial I_(x_)}{\partial y}
\end{bmatrix}
}
% #5 description
{
  Here, the $\nabla I_(x_)$ stands for the gradients of image $I_(x_)$.
}
% #6 Notes
{
This formula is only a brief demonstration to the principle of image processing.As to different images, we need to calculate their gradients on different dimensions and scales.}
% #7 Source
{
  \url{https://en.wikipedia.org/wiki/Image_gradient}
}
% #8 Referenced by
{
  none
}
% #9 Preconditions
{
None
}
% #1 derivation - not applicable by default
{ Not applicable}


~\newline

\subsubsection{General Definitions}\label{sec_gendef}

Not applicable

~\newline



\subsubsection{Input Data Constraints} \label{sec_DataConstraints}    

All the input images must be RGB images with format limited among "JPEG, PNG, BMP, TIFF". 

\section{Requirements}


This section provides the functional requirements, the business tasks that the
software is expected to complete, and the nonfunctional requirements, the
qualities that the software is expected to exhibit.

\subsection{Functional Requirements}

\noindent \begin{itemize}

\item[R\refstepcounter{reqnum}\thereqnum \label{R_OutputInputs}:] Verify the input images must be with the formats as specified in the assumptions.

\item[R\refstepcounter{reqnum}\thereqnum \label{R_Calculate}:] \plt{The RGB images must be transformed into Grayscale image when those images were imported into IFDS system.}

\item[R\refstepcounter{reqnum}\thereqnum \label{R_VerifyOutput}:]
  \plt{The feature points must be displayed onto the grayscale image.}

\item[R\refstepcounter{reqnum}\thereqnum \label{R_Output}:] \plt{Those feature points image must can be saved into user's device.}

\end{itemize}


\subsection{Nonfunctional Requirements}


\noindent \begin{itemize}

\item[NFR\refstepcounter{nfrnum}\thenfrnum \label{NFR_Accuracy}:]
  \textbf{Usability} \plt{The program should be easily understood and usable to end users. This will be tested by random volunteers.}

\item[NFR\refstepcounter{nfrnum}\thenfrnum \label{NFR_Usability}:] \textbf{Portability}
  \plt{This system is developed mainly for Windows OS platform, So it is not assured that it can be used on other OS platforms.}

\item[NFR\refstepcounter{nfrnum}\thenfrnum \label{NFR_Maintainability}:]
  \textbf{Maintainability} \plt{The effort required to make any of the likely
    changes listed for \progname{IFDS} should be less than FRACTION of the original
    development time.  FRACTION is then a symbolic constant that can be defined
    at the end of the report.}


\end{itemize}

\subsection{Rationale}

The assumptions in this documentation have limited the most common format used for images. And those feature points such as corner, edge, BRISK are also typical feature points. Through this system, we meant to showcase those feature points intuitively and visually sensible using classical image processing algorithms. These algorithms include Harris corner points detection, gradients edge points detection, BRISK feature points detection. All those algorithms can be found in literature very easily. 

\section{Likely Changes}    

\noindent \begin{itemize}

\item[LC\refstepcounter{lcnum}\thelcnum\label{LC_meaningfulLabel}:] We may change the image detection algorithms depending on the practical performance. If one method has been proved not feasible, then we may adapt another practical method to realize them.
\item [LC\refstepcounter{lcnum}
\thelcnum\label{LC_meaningfulLabel}:] The layout of GUI may be changed according to user feedback.

\end{itemize}

\section{Unlikely Changes}    

\noindent \begin{itemize}
\item [UC1:] The OpenCV library will not be changed.
\item [UC2:] The coding language will be not changed.
\item [UC3:] The OS platform which it will be distributed to will be changed.
\end{itemize}

\section{Traceability Matrices and Graphs}
The purpose of the traceability matrices is to provide easy references on what
has to be additionally modified if a certain component is changed.  Every time a
component is changed, the items in the column of that component that are marked
with an ``X'' may have to be modified as well.  Table~\ref{Table:A_trace} shows the dependencies of likely changes on
the assumptions.
\afterpage{
\begin{table}[h!]
\centering
\begin{tabular}{|c|c|c|c|}
\hline
	& {A1}& {A2}& {A3} \\
\hline
R1        & & &  \\
\hline
R2       & & &  \\ 

R3       & & & \\ 
\hline
R4           & & &  \\
\hline
LC1           & X& &  \\
\hline
LC2           & & & X \\
\hline
\end{tabular}
\caption{Traceability Matrix Showing the Connections Between Assumptions and other items}
\label{Table:A_trace}
\end{table}
}








\section{Values of Auxiliary Constants}

\plt{Show the values of the symbolic parameters introduced in the report.}

\plt{The definition of the requirements will likely call for SYMBOLIC\_CONSTANTS.
Their values are defined in this section for easy maintenance.}

\plt{The value of FRACTION, for the Maintainability NFR would be given here.}

\newpage

\bibliographystyle{plainnat} 
\bibliography{references} 
\newpage



\newpage{}

\end{document}
